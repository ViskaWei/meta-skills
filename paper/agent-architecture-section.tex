% agent-architecture-section.tex
% Meta-Skills: A Skill Management System for LLM Agents
% Include in main paper — no \documentclass or \begin{document}
%
% Required packages (ensure these are in the main preamble):
%   amsmath, amssymb, booktabs, xcolor, tikz, caption, float, microtype
%   \usetikzlibrary{arrows.meta,positioning,shapes.geometric,fit,backgrounds,calc}

% Color definitions
\definecolor{l0color}{RGB}{52,73,94}
\definecolor{l1color}{RGB}{41,128,185}
\definecolor{l2color}{RGB}{39,174,96}
\definecolor{policyred}{RGB}{192,57,43}
\definecolor{guardpurple}{RGB}{142,68,173}
\definecolor{observeteal}{RGB}{22,160,133}
\definecolor{memoryblue}{RGB}{70,130,180}
\definecolor{arrowgray}{RGB}{120,120,120}
\definecolor{phaseblue}{RGB}{41,128,185}
\definecolor{domainother}{RGB}{120,120,140}


%% ============================================================
%% MAIN BODY (~1.5 pages)
%% ============================================================

\section{Meta-Skills: A Skill Management System}
\label{sec:agent-architecture}

Equipping LLM agents with specialized capabilities creates a scaling dilemma: a library of 200 skill definitions consumes ${\sim}$60{,}000 tokens---30\% of a 200K context window---before a single user message is processed.
Tool selection accuracy degrades significantly once 30--50 tools are active simultaneously~\cite{anthropic-tst-2026,openai-agents-sdk}.
We present \textbf{Meta-Skills}, a skill management system that organizes, discovers, and evolves agent capabilities through three interlocking mechanisms: a \emph{three-layer invocation hierarchy} for progressive disclosure, a \emph{typed contract chain} for automatic pipeline composition, and a \emph{closed-loop evolution mechanism} that transforms every execution into a learning opportunity.

The current system manages 159~atomic capabilities across 12~domains, 25~path templates, and 32~quality policies---while keeping per-query token cost at ${\sim}$4{,}150 tokens ($-91\%$ vs.\ the 47{,}700-token all-load baseline).

\paragraph{Three-layer invocation model.}
The system organizes capabilities into three layers of increasing granularity (Figure~\ref{fig:three-layer}).
\emph{Entry commands} (L0) are 13 user-facing commands plus 12 plugins that accept natural-language requests.
\emph{Path templates} (L1) are 25 declarative YAML recipes specifying multi-step workflows with gate requirements and branch conditions.
\emph{Atomic capabilities} (L2) are 159 contract-driven building blocks with typed inputs and outputs, organized across 12 flat domains.
Cross-cutting \emph{quality policies} (32 rules) inject verification automatically at gate transitions based on artifact types.

\paragraph{Token budget.}
Inspired by Anthropic's Tool Search Tool ($-85\%$ token savings via \texttt{defer\_loading})~\cite{anthropic-tst-2026} and ToolGen's finding that flat indexing outperforms hierarchical (NDCG@1: 87.67 vs.\ 85.67)~\cite{toolgen-iclr-2025}, the system loads capability metadata progressively (Table~\ref{tab:token-budget}).

\begin{table}[ht]
\centering\small
\caption{Token budget by tier.  A typical query (T0+T1.5+T2) costs ${\sim}$4{,}150 tokens, a 91\% reduction from the 47{,}700-token all-load baseline ($159 \times 300$\,tok).}
\label{tab:token-budget}
\begin{tabular}{llrl}
\toprule
\textbf{Tier} & \textbf{Content} & \textbf{Tokens} & \textbf{When loaded} \\
\midrule
T0   & 13 L0 SKILL.md files        & 2{,}600  & Always (auto-discovered) \\
T1   & Full capability catalog      & 5{,}565  & At routing time \\
T1.5 & Domain-filtered catalog      & 350--875 & After L0 narrows domain \\
T2   & Block files (2--3 per query) & 600--1{,}050 & On resolver selection \\
\midrule
\multicolumn{2}{l}{\textbf{Typical total (T0+T1.5+T2)}} & \textbf{$\sim$4{,}150} & \\
\multicolumn{2}{l}{All-load baseline (159 $\times$ 300 tok)} & 47{,}700 & \\
\bottomrule
\end{tabular}
\end{table}

\paragraph{Three original design contributions.}
The architecture introduces three design choices not previously combined in LLM agent skill frameworks (comparison with Anthropic TST, ToolGen, LangGraph, Semantic Kernel, MCP, AutoTool, and CrewAI in Appendix~\ref{app:design-principles}):
\begin{itemize}\setlength\itemsep{2pt}
\item \textbf{Lifecycle as metadata (P4)}: the 8-stage lifecycle (Discover$\to$Knowledge) is preserved as a \texttt{stage:} metadata field rather than a directory structure, enabling capabilities to participate in multiple lifecycle phases.
\item \textbf{Multi-signal type-chain search (P6)}: a 5-signal weighted resolver (trigger~0.30, output\_type~0.25, input\_chain~0.20, domain~0.15, semantic~0.10) replaces single-signal search.
\item \textbf{Typed contract chain (P8)}: 128 artifact types form a type system; the resolver validates $\textit{output\_types}(s_N) \cap \textit{input\_types}(s_{N+1}) \neq \varnothing$, enabling automatic pipeline composition.
\end{itemize}

\paragraph{Completion guard.}
To prevent agents from prematurely terminating multi-step tasks---a persistent failure mode in autonomous LLM systems---the system implements a three-tier completion guard (Appendix~\ref{app:completion-guard}): (1)~a mechanical stop-hook that blocks exit when required evidence is missing, (2)~completion contracts in each SKILL.md declaring required outputs, and (3)~a cross-cutting policy injected into all path templates.

\paragraph{Accompanied execution and self-evolution.}
Every L0 command defaults to a dual-agent team: an \emph{executor} performs the task while a cheaper \emph{observer} model monitors for six signal types---policy violations, capability gaps, step deviations, repeated errors, quality misses, and automation candidates.
Each finding maps to a concrete improvement: a new capability, path, or policy.
Policies function as a \emph{one-way ratchet}: once a failure mode is identified, it is mechanically enforced at every subsequent execution, preventing regression deterministically (Appendix~\ref{app:accompanied-exec}).

\paragraph{Research pipeline.}
The system's most complex instantiation is the \texttt{/research} command, which provides 17 sub-commands, 5 research-specific path templates, and a 3-layer documentation hierarchy (Master Hub $\to$ Topic Hub $\to$ Experiment).
It supports autonomous multi-hour research sessions through a hypothesis-driven agent loop.
Section~\ref{sec:evaluation} evaluates this capability on a complex scientific computing task.

\paragraph{Scope and limitations.}
The core 29-capability infrastructure is open-sourced as a public submodule; domain-specific extensions (130 additional capabilities) are maintained in a private repository.
We evaluate on a single complex task (Section~\ref{sec:evaluation}), which demonstrates depth of capability but does not establish breadth across diverse domains.
The multi-signal resolver weights (P6) are hand-tuned; we leave systematic weight optimization and resolver accuracy evaluation to future work.
The system is built on Claude Code's skill discovery mechanism; portability to other LLM platforms has not been tested.


%% ============================================================
%% APPENDIX
%% ============================================================
%% ============================================================
%% APPENDIX CONTENT
%% NOTE: The parent document must issue \appendix before \input-ing
%% this portion.  For standalone testing, wrap in a test harness.
%% ============================================================
\clearpage

\section{Meta-Skills Architecture Details}
\label{app:agent-architecture}

This appendix provides detailed descriptions for the Meta-Skills system presented in Section~\ref{sec:agent-architecture}.


%% ------------------------------------------------------------
\subsection{System Overview and Key Numbers}
\label{app:overview}

Table~\ref{tab:key-numbers} summarizes the system's current scale.

\begin{table}[ht]
\centering\small
\caption{Meta-Skills system at a glance.}
\label{tab:key-numbers}
\begin{tabular}{lr|lr}
\toprule
\textbf{Component} & \textbf{Count} & \textbf{Component} & \textbf{Count} \\
\midrule
L0 entry commands   & 13 + 12 plugins & Quality policies  & 32 \\
L1 path templates   & 25              & Controlled verbs  & 18 \\
L2 capabilities     & 159             & Governed objects  & 161 \\
Capability domains  & 12              & Artifact types    & 128 \\
\bottomrule
\end{tabular}
\end{table}

\noindent The system's core architecture (FDPS --- Flat-Domain Progressive Skill) is informed by two academic results and an industry consensus: (1)~ToolGen~\cite{toolgen-iclr-2025} showed flat indexing outperforms hierarchical for LLM tool selection (NDCG@1: 87.67 vs.\ 85.67 on 47K tools); (2)~Anthropic TST~\cite{anthropic-tst-2026} demonstrated that progressive disclosure reduces token cost by 85\%; and (3)~all 7 surveyed production frameworks (TST, OpenAI Agents SDK, LangGraph, Semantic Kernel, MCP, CrewAI, AutoGen) organize tools by domain or role, not lifecycle stage.


%% ------------------------------------------------------------
\subsection{Three-Layer Invocation Model}
\label{app:three-layer}

\begin{figure}[ht]
\centering
\begin{tikzpicture}[
    >=Stealth,
    band/.style={rounded corners=5pt, draw=#1!60, fill=#1!6, line width=0.8pt,
        minimum width=13cm, minimum height=1.3cm},
    cmdbox/.style={rounded corners=3pt, draw=l0color!70, fill=l0color!15,
        minimum height=0.48cm, font=\scriptsize\sffamily, inner sep=3pt},
    plugbox/.style={rounded corners=3pt, draw=l0color!40, fill=l0color!5,
        minimum height=0.48cm, font=\scriptsize\sffamily, inner sep=2pt, text=l0color!60},
    pathbox/.style={rounded corners=3pt, draw=l1color!70, fill=l1color!12,
        minimum height=0.48cm, font=\scriptsize\sffamily, inner sep=3pt},
    dombox/.style={rounded corners=2pt, draw=#1!70, fill=#1!18,
        minimum width=1.1cm, minimum height=0.42cm, font=\scriptsize\sffamily, align=center},
    layerlbl/.style={font=\footnotesize\bfseries\sffamily, text=#1},
    sublbl/.style={font=\scriptsize\sffamily, text=black!55},
    vislbl/.style={font=\tiny\sffamily\itshape, text=black!40},
    arrw/.style={->, arrowgray, line width=0.7pt},
]

% === LAYER 0 ===
\node[band=l0color, minimum height=1.7cm] (l0) at (0, 0) {};
\node[layerlbl=l0color, anchor=east] at ([xshift=-4pt]l0.west) {L0};
% Task commands
\node[cmdbox] at (-4.8, 0.3) {/read};
\node[cmdbox] at (-3.5, 0.3) {/write};
\node[cmdbox] at (-2.3, 0.3) {/build};
\node[cmdbox] at (-0.9, 0.3) {/research};
\node[cmdbox] at (0.5, 0.3) {/check};
\node[cmdbox] at (1.6, 0.3) {/fix};
\node[cmdbox] at (3.0, 0.3) {/improve};
\node[cmdbox] at (4.6, 0.3) {/meta};
% Plugins
\node[plugbox] at (-3.0, -0.3) {pdf};
\node[plugbox] at (-1.8, -0.3) {docx};
\node[plugbox] at (-0.5, -0.3) {pptx};
\node[plugbox] at (0.7, -0.3) {browser};
\node[sublbl] at (2.2, -0.3) {\textit{+8 plugins}};
\node[sublbl, anchor=west] at ([xshift=4pt]l0.east) {\textit{13+12}};

% Visibility boundary
\coordinate (visR) at ([xshift=2pt, yshift=-0.32cm]l0.south east);
\coordinate (visL) at ([xshift=-2pt, yshift=-0.32cm]l0.south west);
\draw[black!20, dashed, line width=0.5pt] (visL) -- (visR);
\node[vislbl, anchor=south east] at ([yshift=0.06cm]visR) {visible to user};
\node[vislbl, anchor=north east] at ([yshift=-0.06cm]visR) {\texttt{\_}-prefixed (internal)};

% === LAYER 1 ===
\node[band=l1color] (l1) at (0, -2.2) {};
\node[layerlbl=l1color, anchor=east] at ([xshift=-4pt]l1.west) {L1};
\node[pathbox] at (-4.0, -2.2) {hypothesis-to-evidence};
\node[pathbox] at (-1.0, -2.2) {paper-improve};
\node[pathbox] at (1.5, -2.2) {general-improve};
\node[pathbox] at (3.8, -2.2) {\textit{+22 paths}};
\node[sublbl, anchor=west] at ([xshift=4pt]l1.east) {\textit{25 paths}};

% === LAYER 2 ===
\node[band=l2color, minimum height=1.6cm] (l2) at (0, -4.5) {};
\node[layerlbl=l2color, anchor=east] at ([xshift=-4pt]l2.west) {L2};
% Top row
\node[dombox=l2color] at (-4.8, -4.1) {core (55)};
\node[dombox=l2color] at (-3.0, -4.1) {paper (25)};
\node[dombox=l2color] at (-1.2, -4.1) {research (20)};
\node[dombox=l2color] at (0.6, -4.1) {ml (11)};
\node[dombox=l2color] at (2.2, -4.1) {web (9)};
\node[dombox=l2color] at (3.8, -4.1) {present (7)};
% Bottom row
\node[dombox=domainother] at (-4.8, -4.85) {infra (7)};
\node[dombox=domainother] at (-3.0, -4.85) {knowl. (6)};
\node[dombox=domainother] at (-1.2, -4.85) {media (5)};
\node[dombox=domainother] at (0.6, -4.85) {astro (5)};
\node[dombox=domainother] at (2.2, -4.85) {skill (5)};
\node[dombox=domainother] at (3.8, -4.85) {deliver (4)};
\node[sublbl, anchor=west] at ([xshift=4pt]l2.east) {\textit{159 caps}};

% === ARROWS ===
\draw[arrw] (0, -0.85) -- node[right, sublbl] {\textit{keyword match}} (0, -1.55);
\draw[arrw] (0, -2.85) -- node[right, sublbl] {\textit{multi-signal search (P6)}} (0, -3.7);

% === POLICIES ===
\draw[policyred, line width=1pt, dashed, rounded corners=4pt]
    ([xshift=-2pt, yshift=-0.25cm]l2.south west) rectangle ([xshift=2pt, yshift=-0.68cm]l2.south east);
\node[font=\scriptsize\bfseries\sffamily, text=policyred] at (0, -5.78)
    {33 Quality Policies --- auto-inject at gate transitions by artifact type};

\end{tikzpicture}
\caption{Three-layer skill invocation architecture (FDPS~v4).  25~entry commands (L0) route user requests through 25~path templates (L1) to 159~capabilities in 12~flat domains (L2).  Quality policies inject automatically based on artifact types.}
\label{fig:three-layer}
\end{figure}

\paragraph{Layer~0: Entry commands.}
The system exposes 13 visible commands: 9~task commands (\texttt{/read}, \texttt{/write}, \texttt{/search}, \texttt{/build}, \texttt{/research}, \texttt{/check}, \texttt{/fix}, \texttt{/improve}, \texttt{/capture}), 1~system command (\texttt{/meta}), 2~special commands (\texttt{/workflow}, \texttt{/office}), and 1~schedule command---plus 12~plugin commands.
Each L0 is a thin router: it matches user input via keywords, constrains the domain search space, and delegates to L1 or L2.

\paragraph{Layer~1: Path templates.}
The 25~templates encode expert workflows as YAML recipes spanning six domains: research~(5), paper~(5), general~(6), webui~(3), standards~(3), and docs~(3).
Each specifies capability sequences, gate conditions, branch rules, and policy injection points.

\paragraph{Layer~2: Atomic capabilities.}
The 159~capabilities are organized into 12~flat domains (Table~\ref{tab:domains}).
Each declares typed inputs and outputs from 128~artifact types, enabling the resolver to verify pipeline compatibility.
Names follow a strict \texttt{cap-<verb>-<object>} convention (18~controlled verbs, 161~governed objects) enforced by validation scripts.

\begin{table}[ht]
\centering\small
\caption{Capability domains.  Domain organization follows an industry pattern: all 7 surveyed frameworks organize tools by domain or role rather than lifecycle stage.}
\label{tab:domains}
\begin{tabular}{lrl|lrl}
\toprule
\textbf{Domain} & \textbf{Caps} & \textbf{Scope} & \textbf{Domain} & \textbf{Caps} & \textbf{Scope} \\
\midrule
core       & 55 & Cross-domain fundamentals & media      &  5 & Image, schematic \\
paper      & 25 & Academic writing          & astro      &  5 & Domain-specific   \\
research   & 20 & Hypothesis, experiment    & skill      &  5 & Skill system tools \\
ml         & 11 & ML pipeline               & knowledge  &  6 & Capture, indexing  \\
web        &  9 & Dashboard, explorer       & deliver    &  4 & Package, release   \\
infra      &  7 & GPU setup, observability   & present    &  7 & Slides, poster     \\
\bottomrule
\end{tabular}
\end{table}

\paragraph{Scaling prediction.}
Because domain filtering reduces the catalog search space from $N$ to ${\sim}20$ regardless of total capability count, FDPS token cost grows sub-linearly: at 500~capabilities, predicted cost is ${\sim}5{,}650$ tokens ($-96\%$ vs.\ all-load); at 1{,}000, ${\sim}7{,}400$ tokens ($-98\%$).


%% ------------------------------------------------------------
\subsection{Design Principles}
\label{app:design-principles}

The architecture is grounded in 10~design principles synthesized from 7~frameworks (production systems and research prototypes) and 5~academic papers.
Table~\ref{tab:principles} summarizes all~10; we detail the three novel combinations ($\star$) below.

\begin{table}[ht]
\centering\small
\caption{Ten design principles.  $\star$ = novel combination for LLM agent skill management, drawing on established techniques from information retrieval and dataflow systems.}
\label{tab:principles}
\begin{tabular}{clll}
\toprule
& \textbf{Principle} & \textbf{Primary Source} & \textbf{Implementation} \\
\midrule
P1 & Progressive Disclosure   & Anthropic TST ($-85\%$)           & T0/T1/T1.5/T2 tiers \\
P2 & Flat over Hierarchical   & ToolGen (ICLR 2025)               & Flat \texttt{cap-*} index \\
P3 & Domain-Based Org.        & 7/7 framework survey              & 12 flat domains \\
$\star$P4 & Lifecycle as Metadata & \textit{Original}              & \texttt{stage:} field in catalog \\
P5 & Single Source of Truth   & DRY principle                      & Unified catalog YAML \\
$\star$P6 & Multi-Signal Search   & \textit{Original}              & 5-signal weighted resolver \\
P7 & Soft Ceiling + Delegation & OpenAI $\leq$20, TST $\leq$50   & 13 L0 $\times$ $\sim$15 caps \\
$\star$P8 & Typed Contract Chain  & \textit{Original}              & 128 artifact types \\
P9 & Path Templates           & AutoTool inertia                   & 25 declarative YAML \\
P10 & Governed Extensibility  & MCP + 26.1\% vuln rate            & 18 verbs, 161 objects \\
\bottomrule
\end{tabular}
\end{table}

\paragraph{P4: Lifecycle as metadata.}
The Phase~2 architecture organized capabilities by lifecycle stage, requiring the LLM to first determine which stage it was in.
No existing framework uses lifecycle stage as a primary organization axis; all use domain or role.
FDPS preserves the lifecycle's orchestration value (step ordering, gate transitions, quality-gate injection) as a \texttt{stage:} metadata field, while organizing capabilities by domain.

\paragraph{P6: Multi-signal type-chain search.}
Existing discovery mechanisms use at most two signals: TST uses regex or BM25; LangGraph uses embedding similarity; Semantic Kernel uses contextual RAG.
Our resolver fuses five weighted signals:

\begin{center}\small
\begin{tabular}{lrl}
\toprule
\textbf{Signal} & \textbf{Weight} & \textbf{Provenance} \\
\midrule
Trigger-word match  & 0.30 & TST regex \\
Output-type match   & 0.25 & FDPS original \\
Input-chain compat. & 0.20 & FDPS original \\
Domain match        & 0.15 & Industry consensus \\
Semantic similarity & 0.10 & LangGraph BigTool \\
\bottomrule
\end{tabular}
\end{center}

\noindent The input-chain signal---``which capabilities accept the artifact types the previous step produced?''---enables automatic pipeline step discovery and is unique to FDPS.

\paragraph{P8: Typed contract chain.}
Each capability declares \texttt{input\_types} and \texttt{output\_types} from a vocabulary of 128~artifact types.
The resolver validates $\textit{output\_types}(s_N) \cap \textit{input\_types}(s_{N+1}) \neq \varnothing$ before assembling multi-step pipelines.
No surveyed framework implements inter-tool type contracts: TST, BigTool, and MCP treat tools as independent entities.

\paragraph{Feature comparison.}
Table~\ref{tab:comparison} compares FDPS with seven production frameworks.

\begin{table}[ht]
\centering\small
\caption{Feature comparison. $\bullet$ = full, $\circ$ = partial, -- = absent.}
\label{tab:comparison}
\begin{tabular}{l*{8}{c}}
\toprule
\textbf{Feature} & \rotatebox{65}{\textbf{TST}} & \rotatebox{65}{\textbf{ToolGen}} & \rotatebox{65}{\textbf{BigTool}} & \rotatebox{65}{\textbf{Sem.~K.}} & \rotatebox{65}{\textbf{MCP}} & \rotatebox{65}{\textbf{AutoTool}} & \rotatebox{65}{\textbf{CrewAI}} & \rotatebox{65}{\textbf{FDPS}} \\
\midrule
Progressive disclosure    & $\bullet$ & --        & $\circ$ & $\circ$ & --      & --        & --      & $\bullet$ \\
Flat indexing             & $\bullet$ & $\bullet$ & $\bullet$ & $\circ$ & $\circ$ & --      & --      & $\bullet$ \\
Multi-signal search       & $\circ$   & --        & $\circ$ & $\circ$ & --      & --        & --      & $\bullet$ \\
Typed contract chain$^\star$ & --     & --        & --      & --      & --      & --        & --      & $\bullet$ \\
Path templates            & --        & --        & --      & --      & --      & $\circ$   & --      & $\bullet$ \\
Governed extensibility    & --        & --        & --      & $\circ$ & $\circ$ & --        & --      & $\bullet$ \\
Lifecycle as metadata$^\star$ & --    & --        & --      & --      & --      & --        & --      & $\bullet$ \\
\bottomrule
\end{tabular}
\end{table}


%% ------------------------------------------------------------
\subsection{Completion Guard}
\label{app:completion-guard}

A persistent failure mode in LLM agents is premature task termination: the agent produces partial results and declares completion without verification or delivery.
The completion guard addresses this through a three-tier enforcement architecture introduced in Phase~4.

\paragraph{Tier~1: Mechanical enforcement (stop-hook).}
A shell hook intercepts every session termination attempt and checks a state file tracking completed steps, remaining steps, and required evidence items.
The hook's decision chain:
\begin{enumerate}\setlength\itemsep{1pt}
\item No state file $\to$ allow exit.
\item Iteration $\geq$ max\_iterations $\to$ allow (prevent infinite loops).
\item Transcript contains \texttt{<promise>ALL\_STEPS\_COMPLETE</promise>} $\to$ allow.
\item All steps completed AND all required evidence found $\to$ allow.
\item Otherwise $\to$ \textbf{block exit}, inject system message listing missing items.
\end{enumerate}

\paragraph{Tier~2: Completion contracts.}
Each of the 12~L0 SKILL.md files declares a completion contract specifying required output artifacts.
For example, \texttt{/improve} requires a gate-verdict, improvement-log, and improvement-review with $\geq$4.0/6.0 quality score; \texttt{/research} requires an evidence-bundle and hypothesis-verdict.
Contracts include PAUSE-recovery rules: when execution resumes after user confirmation, the agent must re-check the evidence checklist and continue from the next incomplete item.

\paragraph{Tier~3: Policy injection.}
The policy \texttt{rule-completion-guard} is injected into all 25~path templates.
Each template's \texttt{completion\_guard:} section maps specific evidence items to the workflow (e.g., research paths require \texttt{[evidence-bundle, gate-verdict, knowledge-card]}).


%% ------------------------------------------------------------
\subsection{Accompanied Execution and Self-Evolution}
\label{app:accompanied-exec}

Accompanied execution is a cross-cutting execution mode enabled by default for all L0 commands (opt-out: \texttt{--solo} or \texttt{--depth fast}).
Table~\ref{tab:ae-evolution} summarizes the v1$\to$v2 evolution.

\begin{table}[ht]
\centering\small
\caption{Accompanied execution evolution (Phase~4 v1 $\to$ Phase~5 v2).}
\label{tab:ae-evolution}
\begin{tabular}{lll}
\toprule
\textbf{Dimension} & \textbf{v1 (Phase~4)} & \textbf{v2 (Phase~5)} \\
\midrule
Observation       & 4-dimension full coverage       & 6 high-value signal types \\
Observer model    & Same as executor                 & Haiku ($-70\%$ cost) \\
Cost overhead     & $\sim$2$\times$                  & 1.05--1.5$\times$ (tiered) \\
Complexity tiers  & None                             & Light vs.\ Full \\
Failure handling  & None                             & Graceful degradation \\
Escalation        & None                             & Critical $\to$ leader alert \\
Trend tracking    & Ad-hoc                           & \texttt{PATTERNS.md} index \\
\bottomrule
\end{tabular}
\end{table}

\paragraph{Six signal types.}
The observer monitors for: (1)~policy violations, (2)~capability gaps, (3)~step deviations, (4)~repeated errors ($\geq$2 occurrences), (5)~quality misses, and (6)~automation candidates (manual patterns that could become new path templates).
Each finding maps to a concrete change: a new L2 capability, an L1 path adjustment, or a new \texttt{rule-*} policy.

\paragraph{Policy ratchet.}
Once a failure mode is identified, it is formalized into a policy and enforced mechanically at every subsequent execution.
Because policies trigger on artifact types rather than agent memory, they prevent regression deterministically.
Over the system's lifetime, the policy set grew monotonically from 0 to 32, each addition driven by an observed failure rather than speculative design.

\paragraph{Three-layer memory.}
Cross-session continuity uses a write-on-end protocol: Layer~1 (\texttt{MEMORY.md}, auto-injected); Layer~2 (\texttt{HANDOFF.md}, read at session start); Layer~3 (permanent session archive).
All three must be updated before session termination.

\begin{figure}[ht]
\centering
\begin{tikzpicture}[
    >=Stealth,
    actnode/.style={rounded corners=4pt, draw=phaseblue!60, fill=phaseblue!7,
        minimum width=2.4cm, minimum height=0.7cm,
        font=\small\sffamily, align=center, line width=0.9pt},
    monnode/.style={rounded corners=4pt, draw=observeteal!60, fill=observeteal!7,
        minimum width=2.4cm, minimum height=0.7cm,
        font=\small\sffamily, align=center, line width=0.9pt},
    mechbox/.style={rounded corners=3pt, draw=#1!50, fill=#1!6,
        minimum height=0.5cm, inner sep=5pt,
        font=\footnotesize\sffamily, align=center, line width=0.6pt},
    annot/.style={font=\scriptsize\sffamily, text=black!50},
    primary/.style={->, black!70, line width=1.0pt, shorten >=1pt, shorten <=1pt},
    feedbackedge/.style={->, black!35, line width=0.7pt, dotted, shorten >=1pt, shorten <=1pt},
    crossedge/.style={->, black!25, line width=0.5pt, dashed},
]

\node[actnode] (exec) {Execute};
\node[monnode, right=2.5cm of exec] (observe) {Observe};
\node[monnode, below=1.5cm of observe] (review) {Review};
\node[actnode, below=1.5cm of exec] (evolve) {Evolve};

\begin{scope}[on background layer]
    \node[fit=(exec)(observe)(review)(evolve),
        rounded corners=8pt, fill=black!2, draw=black!8, line width=0.3pt, inner sep=14pt] {};
\end{scope}

\draw[primary] (exec) -- node[above, annot] {6 signals} (observe);
\draw[primary] (observe) -- (review);
\draw[primary] (review) -- node[below, annot] {approve} (evolve);
\draw[feedbackedge] (evolve.west) -- ++(-0.8, 0) |- (exec.west);

\node[annot, above=0.2cm of exec] {executor (Sonnet)};
\node[annot, above=0.2cm of observe] {observer (Haiku)};
\node[annot] at ($(evolve.south east)!0.5!(review.south west) + (0, -0.2)$)
    {$\to$ new cap / path / policy};

\node[mechbox=policyred] (pol) at ($(evolve.south) + (0, -1.6)$) {Policy Ratchet};
\node[mechbox=memoryblue] (mem) at ($(review.south) + (0, -1.6)$) {3-Layer Memory};
\node[mechbox=guardpurple] (guard) at ($(pol)!0.5!(mem) + (0, -0.9)$) {Completion Guard};

\node[annot, anchor=north] at ([yshift=-0.05cm]pol.south) {failure $\to$ rule $\to$ enforce};
\node[annot, anchor=north] at ([yshift=-0.05cm]mem.south) {L1 auto $\cdot$ L2 handoff $\cdot$ L3 archive};

\draw[crossedge] (evolve) -- (pol);
\draw[crossedge] (review) -- (mem);
\draw[crossedge] (pol) -- (guard);
\draw[crossedge] (mem) -- (guard);

\end{tikzpicture}
\caption{Self-evolution loop.  Every command spawns executor + observer.  The observer detects improvement signals; findings feed into new capabilities, paths, or policies via a one-way policy ratchet.  A 3-layer memory ensures cross-session continuity; the completion guard prevents premature termination.}
\label{fig:evolution-loop}
\end{figure}


%% ------------------------------------------------------------
\subsection{Research Pipeline Integration}
\label{app:research-pipeline}

The \texttt{/research} command instantiates the architecture for hypothesis-driven scientific workflows.

\paragraph{17 sub-commands} span the research lifecycle:
\emph{Planning} (\texttt{status}, \texttt{next}, \texttt{lookup});
\emph{Initialization} (\texttt{start}, \texttt{new}, \texttt{init}, \texttt{grow});
\emph{Execution} (\texttt{rq}, \texttt{update}, \texttt{archive}, \texttt{prompt}, \texttt{merge});
\emph{Knowledge} (\texttt{card}, \texttt{design}, \texttt{session});
\emph{Delivery} (\texttt{slides}, \texttt{broadcast}).

\paragraph{5 path templates} encode increasingly comprehensive workflows:
\texttt{new-experiment} (6~steps),
\texttt{knowledge-card} (5~steps),
\texttt{hub-synthesis} (5~steps),
\texttt{ablation-study} (9~steps),
and \texttt{hypothesis-to-evidence} (26~steps with 4~decision branches).

\paragraph{3-layer documentation hierarchy.}
A \emph{Master Hub} provides cross-topic overview with consensus tables.
\emph{Topic Hubs} track problem trees and authoritative numbers for each research question.
\emph{Experiment Reports} document individual runs with front-loaded conclusions and reproducibility commands.
Key findings flow upward: experiment updates synchronize into parent Topic Hub consensus tables.

\paragraph{Autonomous agent loop (\texttt{rq}).}
The \texttt{rq} sub-command launches a fully autonomous research agent that iterates multiple MVP cycles.
The loop implements three feedback channels: \emph{fail-to-fix}, \emph{iterate}, and \emph{pivot}.
Five guardrails were added through eval-driven improvement:
(1)~anti-fabrication (verify claims against actual outputs),
(2)~implementation drift prevention (re-check running code after modifications),
(3)~FINALIZE front-loading (pre-load completion requirements),
(4)~algorithm verification (verify critical implementations against specs),
(5)~MVP-0 sanity gate (establish working baseline first).

\paragraph{Research-specific policies.}
\texttt{rule-research-front-loading} requires the first 30~lines of every experiment report to contain the core conclusion and reproduction recipe.
\texttt{rule-research-eval-protocol} freezes evaluation metrics within a topic.
\texttt{rule-research-reproducibility} mandates seed, environment, and data-version recording.


%% ============================================================
%% BIBLIOGRAPHY ENTRIES
%% Move these to the parent document's .bib file or thebibliography.
%% Required keys: anthropic-tst-2026, toolgen-iclr-2025, openai-agents-sdk,
%%   autotool-2025, agent-skills-wild-2026, agntcy-ads-2025,
%%   registry-evolution-2025, piano, paperbench, scienceagentbench,
%%   mlrbench, masft
%% ============================================================
